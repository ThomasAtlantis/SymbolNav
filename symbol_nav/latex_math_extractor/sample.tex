\documentclass{article}

\begin{document}

\section{Introduction}

This is a test document for the math extractor lexer. Here are various math expressions:

\subsection{Inline Math}

The quadratic formula is $x = \frac{-b \pm \sqrt{b^2 - 4ac}}{2a}$ which solves quadratic equations.

Einstein's mass-energy equivalence is given by $E = mc^2$.

Simple inline math: $a + b = c$ and $x^2 + y^2 = z^2$.

\subsection{Display Math}

The Pythagorean theorem:

$$a^2 + b^2 = c^2$$

A more complex equation:

$$       E = mc^2
$$ 

Another display equation:
$$
\int_0^\infty e^{-x^2} dx = \frac{\sqrt{\pi}}{2}
$$

\subsection{Parenthesis Math}

Using LaTeX's \(x^2 + y^2 = r^2\) notation for inline math.

Another one \(E = mc^2\) here.

Complex expression \(
\int_0^\infty \frac{\sin(x)}{x} dx = \frac{\pi}{2}
\) over multiple lines.

\subsection{Bracket Math}

Display math using brackets:

\[
\sum_{n=1}^\infty \frac{1}{n^2} = \frac{\pi^2}{6}
\]

Another example:
\[
\lim_{x \to \infty} \left(1 + \frac{1}{x}\right)^x = e
\]

\subsection{Equation Environment}

The gamma function:

\begin{equation}
\Gamma(z) = \int_0^\infty t^{z-1} e^{-t} dt
\end{equation}

The famous Euler's identity:

\begin{equation}
e^{i\pi} + 1 = 0
\end{equation}

A more complex equation:
\begin{equation}
  \int_0^\infty e^{-x^2} dx = \frac{\sqrt{\pi}}{2}
\end{equation}

\subsection{Mixed Content}

Here is \textbf{bold text} with $inline math$ and more text.

Using commands like \textit{italic} and $x = y + z$ together.

Multiple inline expressions: $a$, $b$, and $c$ in sequence.

Display math $$E = mc^2$$ followed by inline \(x^2\) and more text.

\subsection{Special Cases}

Math with backslashes: $x \backslash y = z$.

Math with escaped characters: $x^\{2\}$.

Complex nested expressions: $f(x) = \sum_{i=0}^{n} \binom{n}{i} x^i (1-x)^{n-i}$.

Display math with spaces:
$$    x   =   y   +   z    $$

Parenthesis math \(
    x = y
\) on separate lines.

\end{document}

